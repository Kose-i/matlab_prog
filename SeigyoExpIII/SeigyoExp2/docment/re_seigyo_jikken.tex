\documentclass[12pt]{jsarticle}

\setcounter{secnumdepth}{3}
%\setcounter{figure}{1}
\usepackage[dvipdfmx]{graphicx}
\usepackage{amsmath}
\usepackage{float}
\usepackage{bm}
\usepackage{listings}

\pagestyle{myheadings}

\usepackage{titlesec}
\usepackage{cases}
\titleformat*{\section}{\large}
\titleformat*{\subsection}{\normalsize\bf}
\setcounter{equation}{25}
\setcounter{section}{5}

\begin{document}
台車の等価質量$M$[kg],レールとの間の摩擦係数$F$[N]は次のように求められる.
\begin{eqnarray}
  \label{StepResponse_calculate_M}
  M &=& \frac{\alpha h}{a_0} \\
    &=& 8.12057 \nonumber
\end{eqnarray}
\begin{eqnarray}
  \label{StepResponse_calculate_F}
  F &=& M a_1\\
    &=& 34.82427 \nonumber
\end{eqnarray}
また,時間応答に用いるパラメータは次のように求められる.
\begin{eqnarray}
  \label{StepResponse_calculate_omega_n}
  \omega_n &=& \sqrt{a_0} \\
           &=& 4.973897 \nonumber
\end{eqnarray}
\begin{eqnarray}
  \label{StepResponse_calculate_zeta}
  \zeta &=& \frac{a_1}{2\sqrt{a_0}} \\
        &=& 0.431091 \nonumber
\end{eqnarray}
\begin{eqnarray}
  \label{StepResponse_calculate_beta}
  \beta &=& \zeta \omega_n \\
        &=& 2.144202 \nonumber
\end{eqnarray}
\begin{eqnarray}
  \label{StepResponse_calculate_gamma}
  \gamma &=& \omega_n\sqrt{1-\zeta^2} \\
         &=& 4.487990 \nonumber
\end{eqnarray}
\begin{eqnarray}
  \label{StepResponse_calculate_delta}
  \delta &=& {\rm tan}^{-1}\left(\frac{\zeta}{\sqrt{1-\zeta^2}}\right) \\
         &=& 0.445702 \nonumber
\end{eqnarray}
これより,時間応答は次のようになる.
\begin{eqnarray}
  \label{Response_y_t}
  y(t) &=& y_0\left(1 - \frac{1}{\sqrt{1-\zeta^2}}e^{-\beta t}{\rm cos}(\gamma t - \delta)\right) \\
       &=& 0.394657\left(1 - 1.22827e^{-2.1442 t}{\rm cos}(4.48799 t - 0.445701)\right) \nonumber
\end{eqnarray}
実験データと求められた時間応答$y(t)$の相対誤差の平均を式(\ref{Error-mean})を用いて求める.ただし,$x(t)$は実験データを表す.誤差の平均は$3.1668{\rm [\%]}$となった.
\begin{equation}
  \label{Error-mean}
  E = \frac{1}{26}\Sigma_{t=0}^{t=2.5}\frac{|y(t) - x(t)|}{y(t)}
\end{equation}

\clearpage
\setcounter{equation}{41}
\newpage
状態方程式より可制御性についての特性方程式は
\begin{eqnarray}
  \label{regulator-poll-equation}
  {\rm det}(s{\bm I} - ({\bm A}-{\bm b}{\bm f}))&=&s(s+5.1616+2.5924f_2)+2.5924f_1 \nonumber \\
  &=& s^2 + (5.1616+2.5924f_2)s + 2.5924f_1
\end{eqnarray}
であり,可観測性についての特性多項式は
\begin{eqnarray}
  \label{observer-poll-equation}
  {\rm det}(s{\bm I} - ({\bm A}-{\bm K}{\bm c}))&=& (s+k_1)(s+5.1616)+k_2\nonumber \\
  &=& s^2 + (5.1616+k_1)s + (5.1616k_1 + k_2)
\end{eqnarray}
である.
極配置法による可制御性についての特性多項式は
\begin{equation}
  \label{lambda=-10regletor}
  (s - \lambda_1)(s - \lambda_2)=s^2-(\lambda_1+\lambda_2)s+\lambda_1 \lambda_2
\end{equation}
であり,可観測性についての特性多項式は
\begin{equation}
  \label{lambda=-10observer}
  (s - \lambda_3)(s - \lambda_4)=s^2-(\lambda_3+\lambda_4)s+\lambda_3 \lambda_4
\end{equation}
である.
これらの式より,係数比較を行うことにより,次の関係式を示すことができる.
\begin{numcases}
  {}
  f_1 = -\frac{\lambda_1 \lambda_2}{2.5924} & \\
  f_2 = -\frac{\lambda_1 + \lambda_2 - 5.1616}{2.5924} &
\end{numcases}

\begin{numcases}
  {}
  k_1 = - \lambda_3 - \lambda_4 - 5.1616 & \\
  k_2 = \lambda_3 \lambda_4 - 5.1616k_1 &
\end{numcases}
これらの式を用いて実験に用いるパラメータを求め,表\ref{TableB1-2}にまとめた.
\begin{table}[H]
  \begin{center}
    \label{TableB1-2}
    \caption{制御実験に使用したパラメータ}
    \begin{tabular}{|c|c|c|c|c|c|c|c|c|c|} \hline
      実験番号 & $\lambda_1$ & $\lambda_2$ & $\lambda_3$ & $\lambda_4$  & $f_1$ & $f_2$ & $k_1$ & $k_2$ & 図\\ \hline \hline
      1 & $-10$ & $-10$ & $-10$ & $-10$ & $38.5743$  & $5.7238$  & $14.8384$  & $23.4101$ & 13-15 \\ \hline
      2 & $-30$ & $-30$ & $-10$ & $-10$ & $347.1686$ & $21.1535$ & $14.8384$  & $23.4101$  & 16-18\\ \hline
      3 &$-5$  & $-5$  & $-10$ & $-10$ & $9.6436$   & $1.8664$  & $14.8384$  & $23.4101$   & 43-45\\ \hline
      4 &$-20$ & $-20$ & $-20$ & $-20$ & $154.2972$ & $13.4387$ & $34.8384$  & $220.1781$  & 46-48\\ \hline
      5 &$-20$ & $-20$ & $-30$ & $-30$ & $154.2972$ & $13.4387$ & $54.8384$  & $616.9461$  & 49-51\\ \hline
    \end{tabular}
  \end{center}
\end{table}

\newpage

\section{考察}
 実験番号5について,フィードバックゲインの極の位置を$-20$より小さくすると,定常応答の振動が見られる.これは,目標値に対して,操作量が大きくなるためである.
 しかし,実験番号4,5を比較すると,実験番号2は実験番号5に比べ,オブザーバの極が実軸に近いため,入力電圧が大きく変化せず,定常応答の振動が見られなかった.
 実験番号1,3を比較すると,3の入力電圧の最大値は小さくなることが分かる.また,速度についても小さな結果となった.
 実験番号1,2を比較すると,整定時間はあまり変わらない結果となった.これは入力電圧の最大値が10[V]であるためである.
\section{課題}
\begin{description}
  \item[(2)] 制御結果について考察し,収束速度を上げる.過渡応答の振動を抑える,もしくは入力量を抑えるような結果を得るためには,制御系をどのように改良すればよいか検討せよ.\\
  極の値だけで制御されたシステムの挙動が決まるわけではないが,極の値を選ぶための一般的な指針は以下の通りである.極の実部を負の大きな値に取れば,目標値への収束を速くすることができる.ただし,その分,ゲイン$K$が大きくなり,必要な入力$u(t)$が増大することが多い.一方,極の虚部によって,状態$x(t)$の振動的応答を操作できる.実部の大きさに対して虚部の大きさを小さくすることで過大なオーバーシュートを抑制できるが,ある程度の大きさの虚部を与えることで,出力など一部の状態変数の応答を速くすることが可能である\cite{bibitem_pendrol}.
  そのため,収束速度を上げるためには,オブザーバの極とフィードバックゲインの極の値を大きくすることが必要である.ただし,オブザーバの極の実部がフィードバックゲインの極の実部より小さくなければならない.過渡応答の振動を抑えるためには,フィードバックゲインの極とオブザーバの極の虚部が小さくなければならない.入力量の最大値を抑えるためには,フィードバックゲインの極の実部が実軸の近傍になければならない.
\end{description}

\begin{thebibliography}{9}
  \bibitem{bibitemMechatoronics} 坂本哲三,電気機器の電気力学と制御,p157,p158,2018.
  \bibitem{bibitemGendai} 劉康志ら,現代制御理論通論,p121,2006.
%  \bibitem{bibitemFirstGendai} はじめての現代制御p161,p162
  \bibitem{bibitem_pendrol} 川田昌克ら,倒立振子で学ぶ制御工学,p88,2017.
\end{thebibliography}

\end{document}

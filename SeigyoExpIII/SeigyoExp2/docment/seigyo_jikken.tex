\documentclass[12pt]{jsarticle}

\setcounter{secnumdepth}{3}
%\setcounter{figure}{1}
\usepackage[dvipdfmx]{graphicx}
\usepackage{amsmath}
\usepackage{float}

\pagestyle{myheadings}

\usepackage{titlesec}
\titleformat*{\section}{\large}
\titleformat*{\subsection}{\normalsize\bf}

\begin{document}
\section{目的}
状態フィードバック制御を行うためには全状態量をフィードバックする必要がある.しかしセンサの不足などによって実際に測定できない状態量が存在する場合には,オブザーバ(状態観測器)を用いて状態量を推定し,フィードバック制御に利用する.本実験では,台車の位置制御実験を通じて,オブザーバによる状態フィードバック制御について理解を深める.
\section{実験装置の概要}
制御系の構成の概要を図\ref{B2-1}に示す.制御目的は,レール上の台車を目的位置に移動させ,停止させることである.モータドライバに電圧を加えるとモータが回転し,ベルトで台車を移動させる.台車の変位$y$はポテンシャルメータで電圧値として検出され,A/D変換器を介して制御用計算機に送られる.計算機でモータドライバへの指令電圧$u$が計算され,D/A変換を介して出力される.
\section{原理}
\subsection{数式モデル}
モータの回転による力$T_u$を入力,台車の変位$y$を出力としたとき,運動方程式は次のようになる.
\begin{equation}
  \label{Equation-B2-1}
  M\ddot{y} + F\dot{y} = T_u
\end{equation}
$M$は台車の等価質量,$F$はレールとの間の摩擦係数である.$T_u$はパソコンからの指令電圧$u$に比例する.すなわち,$T_u=\alpha u$.ただし,$\alpha = 20.09{\rm [N/V]}$.\\
状態変数を$x=[x_1 x_2]^T = [y \dot{y}]^T$とおくと,次の状態方程式が得られる.
\begin{eqnarray}
  \label{Equation-B2-2}
  \dot{x} &=& \left[[ 0 1 ][0 -F/M]\right] x + \left[[0][\alpha/M]\right] u \\ \nonumber
  y &=& [1 0] x
\end{eqnarray}
ただし$T$はベクトルの転置を意味する.
\subsection{未知パラメータの同定}
状態フィードバック制御系を構成するためには,式(\ref{Equation-B2-2})の中の動特性パラメータがわかっている必要がある.システムのステップ応答より$F$と$M$を同定することができる.式(\ref{Equation-B2-2})より,台車に一定の力を入力すると,台車は一方向に走りっぱなしになる.そこで,次の出力フィードバックをかける.
\begin{equation}
  \label{Equation-B2-3}
  u = h(r - y)
\end{equation}
ただし,$r$は台車の目標ステップ変位,$h$はフィードバックゲイン定数である.すると,$r$から$y$までの伝達関数は式(\ref{Equation-B2-4})となる.この系の応答波形は$h$の値によって変わる.
\begin{equation}
  \label{Equation-B2-4}
  G(s) = \frac{Y(s)}{R(s)} = \frac{b_0}{s^2+a_1s+a_0} (a_1 =F/M,a_0=b_0=\alpha h /M)
\end{equation}
したがって,$a_0$,$a_1$が求まれば,$F$,$M$が計算できる.$a_0$,$a_1$の求め方はステップ応答を用いた二次遅れ系のパラメータ同定,同定結果の検証の節を参照せよ.
\subsection{制御系の構成}
式(\ref{Equation-B2-2})に対し以下のように状態フィードバック制御系を構成する.
\begin{eqnarray}
  \label{Equation-B2-5}
  \dot{x}(t) &=& Ax(t) + bu(t) \\
  \label{Equation-B2-6}
  y(t) &=& cx(t)
\end{eqnarray}
ただし,
\begin{equation}
  \label{Equation-B2-7}
  A = \left[ [0 1] [0 -F/M]\right] b = \left[ [0] [\alpha/M]\right] c = [1 0]
\end{equation}
\begin{equation}
  \label{Equation-B2-8}
  u(t) = -fx(t) = [-f_1 -f_2][x_1(t) x_2(t)]^T
\end{equation}
本実験では二つの状態,すなわち台車位置$x_1(t)$と台車速度$x_2(t)$のうちセンサで検出できるのは$x_1(t)$だけである.したがって,状態フィードバックを行うためには,オブザーバによる状態の推定値を用いる必要がある.そこで,式(\ref{Equation-B2-6})のプラントに対し,次の同一次元オブザーバを構成する.
\begin{eqnarray}
  \label{Equation-B2-9}
  \dot{\hat{x}}(t) &=& A \hat{x} + bu(t) + k(y(t) - \hat{y}(t)) \\
  \label{Equation-B2-10}
  \hat{y}(t) &=& c\hat{x}(t)
\end{eqnarray}
$\hat{x}(t)$は状態$x(t)$の推定状態量である.これらより,台車を初期位置から目標位置に移動させる制御は次の入力を用いることによって達成される.
\begin{equation}
  \label{Equation-B2-11}
  u(t) = -f \hat{x}(t) = [ -f_1 -f_2][(\hat{x}_1 (t) - y_r) \hat{x}_2(t)]^T
\end{equation}
ただし,$y_r$は台車の目標位置である.制御系全体の構成を図\ref{Fig2-2}に示す.
\subsection{ステップ応答を用いた二次遅れ系のパラメータ同定}
\subsection{同定結果の検証}
\subsection{状態フィードバック制御系の設計}
\section{実験方法}
\section{課題}
\end{document}

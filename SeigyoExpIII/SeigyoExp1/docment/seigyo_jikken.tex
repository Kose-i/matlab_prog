\documentclass[12pt]{jsarticle}

\setcounter{secnumdepth}{3}
\usepackage[dvipdfmx]{graphicx}
\usepackage{amsmath}

\begin{document}

%{\LARGE レベル制御(PI制御)}
\section{目的}
 プロセス制御系の制御対象(プロセスまたはプラントともいう)は,一般に,炉,反応装置などの化学プロセスが主である.それらは分布定数系,もしくはむだ時間を含む系であることが多く,その応答時間は数分から数10分にもなり,サーボ系に比べ,極めてゆっくりとしているのが特徴である.ここでは,プロセス系の一つであるレベル制御を最も一般的なPID調節器の設計を通じて学習する.
\section{原理}
\section{実験装置と注意事項}
\subsection{実験装置}
\subsection{プログラム}
\subsection{実験の開始時と終了時の操作}
\section{開ループ実験}
{\LARGE 実験(1) 制御弁}
流量の制御装置として使用している制御弁の動特性を調べた.
\begin{figure}[tb]
  \begin{center}
    \includegraphics[clip,width=7.0cm]{../graph/V_Q_transform.png}
    \caption{}
    \label{V_Q_transform}
  \end{center}
\end{figure}
プログラム「./expapp」を起動した.制御電流を4[mA]から20[mA]まで変化させ,そのときの定常状態の流量を測定し,電流-流量特性を求めた.
{\LARGE 実験(2) 差圧変換器(下段タンク用)}
水位検出器として使用している差圧変換器の動特性を調べた.
\fig{H_V_transform}
プログラム「./expapp」を起動した.制御弁を開閉して,タンク2の水位を増加させながら,そのときの差圧変換器の出力電圧を調べた.タンク2水位-出力電圧特性を求めた.ただし,測定する水位の上限は60[cm]とする.
{\LARGE 実験(3) 開ループ応答}
制御対象であるタンク1およびタンク2にステップ入力を加え,その過渡応答を求めた.
\fig{Q_H_V_transform}
プログラム「./step」を起動した.プログラムの指示に従い,(1)制御弁の初期値,(2)変化量,(3)サンプリング周期を入力した.その後,[APPLY AND START]をクリックして実験を開始した.
まず,制御弁が初期位置まで開き,タンク1に水が流れ込む.タンク2の水位が定常地に収束した後に,[Recording Start]をクリックするとタンク2の差圧変換器の出力の計測が始まる.[Recording Start]をクリックしてから10サンプル時刻後に「変化量」で与えた値だけ更に制御弁が開き,タンク2の差圧変換器出力の計測が始まる.出力値が定常状態に収束したら「STOP」をクリックした.この時,ファイルネーム(英数字)に拡張子(dat)を付けて「SAVE ONLY」をクリックしてデータを保存した.この時のデータの書式は\ref{OpenLoopResponseDataFormat}に示す通りである.データの横軸を時間軸に変換するには,「サンプル番号×サンプリング周期」とすればよい.
\section{モデル同定(第2週)}
各制御要素を次の手順で近似せよ.
\begin{itemize}
\item 実験(1)で電流-流量特性を作図するとともに,作動点近傍で線形化し制御弁のゲインを求め,ブロック線図を描け.
\item 実験(2)で下段タンク水位-出力電圧特性を作図するとともに,線形化し差圧変換器のゲイン(必要ならオフセットも)を求め,ブロック線図を描け.更に,次週で使用するので,出力電圧-水位の関係も求めておくこと.
\item 実験(3)で得られた結果を作図し,タンク1およびタンク2の伝達関数を一次遅れ+むだ時間で近似し,ブロック線図を描け.
\item 実験(1)-(3)で決定した各要素の合成結果と実験結果とを比較検討せよ.
\end{itemize}
\section{PID調節器の調整}
開ループの結果より,Ziegler-Nicholsの限界感度法に基づきPID調節器のパラメータを決定した.なお,限界感度法においてはBode線図を作図して決定すること.本実験では,PI動作のみで制御を行った.つまり,PID調節器のパラメータ$T_D$とし他の$K_P$と$T_I$を調整する.
\fig{FeedBackControlProcess}
\section{閉ループ実験(第3週)}
プログラム「./PID」を起動した.
プログラムの指示に従い,以下のパラメータを入力した.\\
(1)比例ゲイン,(2)積分時間,(3)微分時間,(4)タンク2の目標水位,(5)差圧変換器のゲイン$D_{gain}$及びオフセット$D_{offset}$,(6)サンプリング周期\\
ここで,差圧変換器の出力$D_{out}$[V]と水位$H$[cm]の関係を
\begin{equation}
  D_{out} = H・D_{gain} + D_{offset}
\end{equation}
として計算している.目標水位は40[cm]とした.\\
[APPLY AND START]をクリックして実験を開始した.[Apply disturbance]をクリックすると,外乱要素として排水用制御弁の開度が変化するので,その応答を観測し検討した.600サンプル後もしくは[STOP]ボタンをクリックした後,[SAVE OK]をクリックしてデータを保存し,ファイルネーム(英数字)に拡張子(.dat)を付けて保存した.このときのデータは図\ref{ClosedLoopDataFormat}に示す書式で印刷された.

\section{課題}
\begin{description}
\item[(1)]結果をグラフに描き,応答を考察せよ.\\
\item[(2)]PI調節器+制御対象のボード線図を描き,ゲイン余裕と位相余裕を考察せよ.\\
\item[(3)]PID調節器のパラメータの決め方は,他にどんなものがあるのか調べよ.\\
\end{description}

\section{まとめ}
PID制御に使用するPID調節器の設計を習得した.

\begin{thebibliography}{9}
\end{thebibliography}

\end{document}
